\documentclass{article} 
\usepackage{iclr2022_conference,times}
\input{math_commands.tex}

\newcommand{\apsname}{Project Proposal}
\newcommand{\gpnumber}{34}

\usepackage{hyperref}
\usepackage{url}
\usepackage{graphicx}

%######## APS360: Put your project Title here
\title{Titleeeeeee  \\ 
based on ICLR Conference Format}


%######## APS360: Put your names, student IDs and Emails here
\author{Pengcheng Wang  \\
Student\# 1006769622\\
\texttt{pcheng.wang@mail.utoronto.ca} \\
\And}
\author{Zihao Zhao  \\
Student\# 1007358244 \\
\texttt{mark.zhao@mail.utoronto.ca} \\
\AND}
\author{Ruiqi Zhu  \\
Student\# 1006661676 \\
\texttt{rachelzrq.zhu@mail.utoronto.ca} \\
\And}
\author{Zechen Duan \\
Student\# 1006784936 \\
\texttt{author4@mail.utoronto.ca} \\
\AND
}


\newcommand{\fix}{\marginpar{FIX}}
\newcommand{\new}{\marginpar{NEW}}

\iclrfinalcopy 

\begin{document}


\maketitle

\begin{abstract}
write here\\
----Total Pages: \pageref{last_page}
\end{abstract}


\section{Introduction}

\section{Illustration / Figure}

\section{Background & Related Work}

\section{Data Processing}

\section{Architecture}

\section{Baseline Model}

\section{Ethical Consideration}

\section{Project Plan}

\section{Risk Register}

\begin{tabular}{|c|c|}% 通过添加 | 来表示是否需要绘制竖线
\hline  % 在表格最上方绘制横线
(1,1)&(1,2)\\
\hline  %在第一行和第二行之间绘制横线
(2,1)&(2,2)\\
\hline % 在表格最下方绘制横线
\end{tabular}



\section{Link to Github}

\newpage
\subsubsection*{Acknowledgments}
QNMD

\label{last_page}

\newpage
\bibliography{APS360_ref}
\bibliographystyle{iclr2022_conference}

\end{document}
